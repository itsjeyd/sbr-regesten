\section{Extraction Process: Regesten}
\label{sec:regesten}
%% - Wie ist ein einzelnes Regest aufgebaut?
%% - XML-Schema für einzelne Regesten beschreiben / erklären; mit
%%   (konstruiertem) Beispiel, in dem alles vorkommt
%% - Extraktion beschreiben (control flow, evtl. als Diagramm; Behandlung
%%   von Sonderfällen / Ausnahmen; etc.)

\emph{Author: Conrad Steffens} \\

The ``Regesten'' part of the Sbr-Regesten contains the regest
documents. The XML schema for regests is defined in

\begin{verbatim}
sbr-regesten/regesten-schemas/sbr-regesten.xsd
\end{verbatim}

together with the schema for the other parts of the book. As described
in Chapter \ref{sec:overview}, the module

\begin{verbatim}
sbr-regesten/extraction/regest_extractor.py
\end{verbatim}

(which still needs to be implemented) takes care of extracting
and annotating regests. It uses the HTML version of the Sbr-Regesten
stored in \texttt{sbr-regesten.html} as input. The XML output it
produces is added to \texttt{sbr-regesten.xml}, and a database entry
is created for each individual regest.

Additionally, a postprocessing step for tagging proper names of e.g.
persons and locations should be implemented: Once the index has been
parsed, regest references in the index can be exploited in order to
detect entities mentioned in a given regest, and annotate them using
the \texttt{<name>} tag.
