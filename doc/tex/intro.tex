\section{Introduction}
\label{sec:intro}
%% Kurze Einleitung: Was sind die Sbr Regesten und was haben wir mit
%% ihnen gemacht? Wie ist der Rest der Dokumentation gegliedert?

This documentation accompanies work that was carried out over the
course of nine months by Susanne Fertmann
\footnote{\texttt{s9sufert@stud.uni-saarland.de}}, Tim Krones
\footnote{\texttt{tkrones@coli.uni-saarland.de}}, and Conrad Steffens
\footnote{\texttt{conradsteffens@googlemail.com}} for the Software
Project ``Computerlinguistische Aufarbeitung kulturhistorischer
Dokumente'', under supervision of Dr. Caroline Sporleder. The project
targets the book ``Regesten zur Geschichte der Stadt Saarbrücken (bis
1545)'' by Irmtraut Eder-Stein, which is a collection of historical
deeds (\emph{Urkunden}) pertaining to the history of the city of
Saarbrücken. Throughout this document, we will refer to this work
simply as \emph{Sbr-Regesten}.

The main goal of the project was to come up with a digital
representation of the book in order to lay the foundation for making
the information contained in the original text easily searchable, as
well as for extracting different kinds of implicit knowledge from it.

To achieve this goal, the following (sub-)tasks were identified and
completed within the scope of the project:

\begin{enumerate}
\item Design a custom \emph{XML schema} for the Sbr-Regesten
\item Extract all information contained in the original text and
  annotate it according to this schema
\item Based on the XML schema, create a database schema
\item Store the information extracted from the Sbr-Regesten in a
  database created from this schema
\item Set up a web application on top of the database that allows
  users with administrative rights to browse, add, modify, and delete
  database content
\end{enumerate}

The remainder of this document is structured as follows: Chapter
\ref{sec:overview} provides an overview of the different components
that were implemented for the purpose of extracting, annotating, and
storing information contained in the Sbr-Regesten, and how they are
chained to form a full \emph{extraction pipeline}. Chapters
\ref{sec:regesten} and \ref{sec:index} provide a detailed account of
the XML schema and extraction logic for the Regesten and Index
chapters of the Sbr-Regesten. Chapter \ref{sec:other} explains the XML
mark-up for the remaining parts of the book, and how they were
extracted from the original text. The last chapter focuses on the web
application, providing general information about the
\href{https://www.djangoproject.com/}{Django Web Application
  Framework}, as well detailed guidance for installing, running,
using, deploying and extending the application. It also includes
information about the data model the web application is based on.

Chapter \ref{sec:index} was written by Susanne Fertmann, and chapter
\ref{sec:webapp} was written by Tim Krones. Chapter
\ref{sec:regesten}, as well as sections \ref{sec:preface} and
\ref{sec:archives} will be supplied by Conrad Steffens at a later
date. The remaining chapters represent a joint effort between Susanne
Fertmann and Tim Krones.
