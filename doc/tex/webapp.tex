\section{Web Application}
%% (Reihenfolge noch vorläufig)
%% - Was ist Django?
%% - Wie setzt Django das MVC-Pattern um?
%% - Installation einschließlich Dependencies
%% - Benutzung der App (allgemein und Admin-Interface)
%% - Deployment
%% - Datenmodell (Ähnlichkeiten / Unterschiede zum XML-Schema)
%% - Zukünftige Erweiterungen (Templates und Views für Endnutzer,
%%   XML-Export)

\emph{Author: Tim Krones} \\

In addition to functionality for extracting and annotating the
contents of the Sbr-Regesten, the package also provides the basic
architecture for a web application. This chapter gives an overview of
the Django Framework for developing web application. It then describes
how to

\begin{itemize}
\item install necessary dependencies
\item run the application
\item use the admin interface to add or change data
\end{itemize}

and provides basic pointers for extending and deploying the
application. Also included is a detailed description of the data model
used for storing information extracted from the Sbr-Regesten in the
database associated with the application.

\subsection{The Django Web Framework}
\label{sec:django}

The information in this section is based on the official Django
documentation which can be found at
\url{https://docs.djangoproject.com/}.

Django is a Python Framework for rapid prototyping and development of
interactive web applications. It uses the MVC pattern to separate the
different tasks that are involved in creating interactive web
applications.

\subsubsection{The MVC Pattern}
\label{sec:mvc}

\href{https://en.wikipedia.org/wiki/Model_view_controller}{Model-View-Controller},
or MVC for short, is a
\href{https://en.wikipedia.org/wiki/Software_design_pattern}{Software Design Pattern}

commonly used by Web Frameworks such as Django and Ruby on Rails. The
basic idea of MVC is is to divide application logic into three layers.
The \emph{Model} layer is responsible for storing and operating on
data. This usually involves at least the basic
\href{https://en.wikipedia.org/wiki/CRUD}{CRUD} operations
\emph{Create}, \emph{Read}, \emph{Update}, and \emph{Delete}.The
\emph{View} layer takes care of presenting available data to end
users. \emph{Controllers} are responsible for handling user requests.
Depending on the type of request, this usually involves querying the
model layer for data, manipulating this data in various ways (if
necessary), and sending it off to the view layer for presentation.

Different frameworks interpret MVC in different ways; the next chapter
describes Django's implementation of this pattern.

\subsubsection{How Django Implements MVC}
\label{sec:django-mvc}

This chapter presents an overview of how Django interprets and
implements the MVC pattern. For an in-depth treatment of the
individual components, please consult the documentation at
\url{https://docs.djangoproject.com/}.

While Django's seperation of concerns is heavily influenced by the MVC
pattern conceptually, the framework uses a different terminology to
distinguish the individual components for dealing with (user)
requests, data, and presentation. The terminological differences tend
to confuse users that are new to Django or to working with MVC
frameworks in general, which makes it all the more important to
understand these differences before delving into Django development.

Django distinguishes between \emph{models}, \emph{templates}, and
\emph{views}, which is why the framework is commonly referred to as an
``MTV'' framework. The model layer in Django corresponds to the
concept of a model layer as it is defined (or at least commonly
understood) in the context of MVC. Django templates correspond to
views in MVC, and the responsibilities of Django views are similar to
those of controllers in MVC.

From an architectural point of view, a Django \emph{project} usually
consists of one or more Django \emph{apps}. Among other things, each
app includes a dedicated Python module for the model layer (called
\texttt{models.py}), two Python modules that are jointly responsible for
handling user requests (called \texttt{views.py} and \texttt{urls.py})
and a hierarchy of templates written in Django's template language.

The \texttt{models.py} module contains specialized Python classes
(called \emph{models}) which define the data model of a given Django
app. Each class corresponds to a table in the database of the project,
with additional tables being created as necessary to represent
relationships between different models.

The \texttt{views.py} module contains specialized Python functions
(called \emph{view functions}) for handling user requests. These
functions are responsible for querying the database for information,
manipulating that information if necessary, and rendering the
appropriate templates back to the user, filled with the information
that was requested. In this context, the \texttt{urls.py} module acts
as a kind of \emph{dispatcher}: It contains a mapping from URLs (or,
generally speaking, URL patterns) to appropriate view functions,
allowing Django to identify the actions it needs to take based on the
URL that was requested by the user.

\subsection{Installing and Using the Web Application}
\label{sec:webapp}

This chapter explains how to install and use the Sbr-Regesten Web
Application, and also provides some pointers on how to extend and
deploy it.

\subsubsection{Installation}
\label{sec:install}

\paragraph{Python and Django}
The Sbr-Regesten Web Application was developed using Python 2.7.3 and
Django 1.4.3.

Python binaries and source code for all major operating systems can be
obtained from \url{http://python.org/download/}. Python binaries are
usually pre-installed on Linux distributions, and different versions
can be obtained from standard repositories using a package manager:
Please note that at the time of this writing, Django is \textbf{not}
compatible with Python 3, so in order to run the app successfully,
Python 2.7.* needs to be installed.

The easiest way to install specific versions of Django is using the
\href{http://www.pip-installer.org/en/latest/}{pip-installer} which is
a tool for installing and managing Python packages. \texttt{pip}
should be available in the standard repositories of most Linux
distributions (package: python-pip). For generic installation
instructions, visit
\url{http://www.pip-installer.org/en/latest/installing.html}.

Once \texttt{pip} has been installed, version 1.4.3 of Django can be
installed using the following command:

\begin{verbatim}
$ pip install Django==1.4.3
\end{verbatim}

For instructions on how to install Django manually, consult
\href{https://www.djangoproject.com/download/}{this part} of the
Django documentation.

\paragraph{BeautifulSoup}
The process of extracting and annotating information from the
Sbr-Regesten makes heavy use of a tool called \emph{BeautifulSoup},
which needs to be installed in order to reproduce the extraction
process locally.

Like Django, BeautifulSoup is pip-installable:

\begin{verbatim}
$ pip install beautifulsoup4
\end{verbatim}

For the purpose of improving or extending the extraction process,
detailed information about BeautifulSoup can be found in its
\href{http://www.crummy.com/software/BeautifulSoup/bs4/doc/}{official documentation}.

\paragraph{Further Recommendations}
In addition to the hard dependencies described in the previous
sections, we recommend installing the \emph{IPython} interpreter as it
provides a lot of features not included in the standard python
interpreter and thus makes interacting with the database from Django's
development shell a lot easier. The latest version of IPython can be
installed using pip as follows:

\begin{verbatim}
$ pip install ipython
\end{verbatim}

\subsubsection{Running the Application}
\label{sec:run}

Once all necessary dependencies are installed, you can run the
application. Extract the contents of the source archive to an
appropriate folder in your file system and \texttt{cd} into the root
folder of the project. This folder is called \texttt{sbr-regesten}.
Look for a file called \texttt{sbr-regesten.db}. If it's there, this
means that source package you have received includes a prepopulated
database, and that you can run the application right away by typing

\begin{verbatim}
$ python manage.py runserver
\end{verbatim}

If you find that the database file is missing from the source archive,
you need to proceed as follows: First, initialize the database by running

\begin{verbatim}
$ python manage.py syncdb
\end{verbatim}

At some point you will be asked whether or not you would like to
create a superuser for the database. Type \texttt{yes} and press
Enter, then provide a username, email address and password. For
development purposes it is both convenient and acceptable to simply
set username and password to \texttt{admin}.

When the \texttt{syncdb} command finishes, you can either start using
the application with an empty database by typing the
\texttt{runserver} command listed above, or you can go ahead and
populate the database with information from the Sbr-Regesten. Since
the extraction process was implemented as a Django \emph{management command},
you can trigger it using the following command:

\begin{verbatim}
$ python manage.py extract
\end{verbatim}

Note that this process might take a long time to finish.

\subsection{Django Data Model}
\label{sec:data-model}
